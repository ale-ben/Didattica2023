\documentclass[../../relazione.tex]{subfiles}

\begin{document}
\section{Inquadramento del lavoro}

\subsection{Livello di scuola, classe/i, indirizzo}
Questa unità didattica è progettata per la scuola secondaria di primo grado.

\subsection{Motivazione e Finalità WIP}
La lezione si inserisce nella parte finale di un corso di programmazione con linguaggio a blocchi (si suppone Scratch).

L'obiettivo principale è quello di consolidare le conoscenze acquisite durante il corso applicandole ad un nuovo ambito, quello della programmazione di microcontrollori (nello specifico, Micro:Bit).

TODO: aggiungere motivazione

\subsection{Innovatività TODO}
TODO: aggiungere innovatività

\subsection{Prerequisiti}
Si suppose che gli studenti abbiano già acquisito familiarità con un linguaggio di programmazione a blocchi e che ne siano stati trattati almeno i seguenti argomenti:
\begin{itemize}
	\item cicli while;
	\item variabili di tipo stringa e numerico;
	\item condizioni nei cicli e nelle selezioni;
	\item Input / Output in Scratch;
	\item Funzioni in scratch
\end{itemize}

\subsection{Contenuti}
L'unità si concentra sul trasferimento delle conoscenze acquisite durante il corso di programmazione a blocchi ad un nuovo ambiente, quello della programmazione di microcontrollori.\\
In particolare si approfondirà il concetto di input / output in Micro:Bit tramite l'uso rispettivamente di pulsanti e display integrato. 

\subsection{Traguardi e Obiettivi}

\subsubsection{Collegamento con i documenti ministeriali/proposte}
\paragraph{MIUR}
Facendo riferimento al \textit{Decreto 16 novembre 2012, n. 254, Art. 2} \cite{MIURWeb, MIUR254} emesso dal Ministero dell'Istruzione, dell'Università e della Ricerca, questa unità didattica si inserisce nella disciplina \textit{Tecnologia}.

\paragraph{CINI}
Facendo riferimento alla \textit{Proposta di Indicazioni Nazionali per l'insegnamento dell'Informatica nella Scuola} \cite{CINI} rilasciata dal Consorzio Interuniversitario Nazionale per l'Informatica, pagine da 6 a 8, questa unità didattica contribuisce al raggiungimento dei seguenti traguardi e obiettivi:
\subparagraph{Traguardi}
\begin{itemize}
	\item \textbf{T-M-5} progetta, scrive e mette a punto, usando linguaggi di programmazione facili da usare, programmi che applicano selezione, cicli, variabili e forme elementari di ingresso e uscita;
	\item \textbf{T-M-7} riconosce dati di ingresso e di uscita delle applicazioni informatiche;
\end{itemize}
\subparagraph{Obiettivi}
\begin{itemize}
	\item \textbf{O-M-P-1} sperimentare piccoli cambiamenti in un programma per capirne il comportamento, identificarne gli eventuali difetti, modificarlo;
	\item \textbf{O-M-P-6} usare le variabili nelle condizioni dei cicli e delle selezioni;
	\item \textbf{O-M-N-4} connettere dispositivi informatici tra di loro e con periferiche, anche per realizzare semplici esperienze di raccolta ed analisi dati e di controllo di dispositivi esterni;
\end{itemize}

\subsubsection{Obiettivi di apprendimento}
Facendo riferimento alla tassonomia Bloom rivisitata \cite{BLOOM}, lo studente è in grado di:
\begin{itemize}
	\item Stabilire corrispondenze tra il linguaggio a blocchi noto e quello presentato;
	\item Ipotizzare il comportamento di un programma per Micro:Bit in un determinato scenario e verificare la correttezza delle proprie ipotesi;
	\item Costruire in autonomia un programma a partire dalle specifiche date;
	\item Testare il prorio programma, prima per mezzo di un simulatore e, successivamente, su un dispositivo fisico;
\end{itemize}

\subsection{Metodologie didattiche}

Le lezioni sono composte principalmente da attività di laboratorio alternate a discussioni che stimolano lo studente a formulare e verificare delle ipotesi sul comportamento del sistema.\\
Tutte le attività vengono svolte in gruppi di 2-3 studenti, in modo da favorire il lavoro di gruppo e la collaborazione.\\
L'unità didattica aderisce principalmente al paradigma del costruttivismo, facendo principalmente uso di \textit{Active Learning} e \textit{Project Based Learning}.

\subsection{Tempi}
L'unità didattica dovrebbe richiedere tra le 8 e le 12 ore di lezione, distribuite in 2-3 settimane.\\
Sarebbe ottimale dedicare le prime 2/3 ore alla familiarizzazione con il nuovo ambiente e linguaggio, nonche all'introduzione e alla sperimentazione con Micro:Bit. In questo periodo si possono si può anche introdurre la prima attività.\\
Successivamente si può procedere con la seconda e terza attività, queste dovrebbero richiedere complessivamente tra le 2 e le 3 ore. Al termine di queste ore lo studente dovrebbe aver acquisito sufficiente familiarità con l'ambiente per poter lavorare al progetto.\\
Infine si può procedere con la quarta attività e quinta attività, che dovrebbero richiedere complessivamente tra le 2 e le 4 ore. 

Le ore stimate per le attività risultano essere tra le 6 e le 10, lasciando almeno 2 ore disponibili per le varie discussioni suggerite nel corso delle attività.

\subsection{Spazi}
Tutte le attività prevedono un laboratorio attrezzato con computer con connessione ad internet per ogni gruppo di studenti e per il docente. 

\subsection{Materiali e Strumenti}
Per lo svolgimento delle attività è necessario un computer con connessione ad internet per ogni gruppo di studenti e per il docente.\\
Sono inoltre necessari alcuni microcontrollori Micro:Bit (è utilizzabile anche la versione 1), sarebbe ottimale averne uno per ogni gruppo di studenti ma, dato che l'editor ha un simulatore integrato, è anche accettabile fornirne uno ogni due gruppi chiedendo ai gruppi di condividerne l'uso.

\subsubsection{Versione Micro:Bit}
Salvo diversamente indicato, tutte le attività proposte sono compatibili con entrambe le versioni di Micro:Bit.\\
Questa scelta riduce notevolmente il bacino sensori disponibili e, di conseguenza, di attività che si possono realizzare ma è stata presa al fine di ridurre i costi dei materiali e rendere l'attività più accessibile.

\end{document}