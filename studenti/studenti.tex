\documentclass{beamer}
\usepackage[utf8]{inputenc}
\usepackage[italian]{babel}
%Information to be included in the title page:
\title{Approfondimento sulla programmazione a blocchi con Micro:Bit}

\graphicspath{{./media/}}

\begin{document}

\frame{\titlepage}

\begin{frame}
	\frametitle{Introduzione}
	
	\begin{figure}[h]
		\includegraphics[width=11cm]{mbScheme.png}
	\end{figure}

\end{frame}

\begin{frame}
	\frametitle{Editor}
	
	\begin{center}
		\textbf{https://microbit.org/join}
	\end{center}

	
\end{frame}

\begin{frame}
	\frametitle{Scratch e Make:Code}

	\textit{Quali sono le similitudini e quali le differenze tra Scratch e Make:Code?}

\end{frame}

\begin{frame}
	\frametitle{Attività 1 - Display}
	
	Realizzare un programma che scriva sul display il contenuto di una variabile ti tipo stringa.

	\vspace{2em}
	\begin{enumerate}
		\item Inizializzare la variabile con un valore
		\item Visualizzare il valore
	\end{enumerate}

	\vspace{2em}
	\textit{C'è differenza tra come viene visualizzata una stringa composta da un solo carattere e una stringa composta da più caratteri?}

\end{frame}

\begin{frame}
	\frametitle{Attività 2 - Contatore}

	Realizzare un programma che incrementa o decrementa una variabile di tipo numerico alla pressione di un pulsante e ne visualizza il valore sul display.

	\vspace{2em}
	\begin{itemize}
		\item Il tasto A deve decrementare la variabile di 1
		\item Il tasto B deve incrementare la variabile di 1
		\item (EXTRA) Il valore della variabile deve sempre rimanere compreso tra 0 e 9
	\end{itemize}
	

\end{frame}

\begin{frame}
	\frametitle{Attività 3 - Cronometro}

	Realizzare un cronometro che misuri il tempo trascorso tra la prima e la seconda pressione del pulsante A.

	\vspace{2em}
	\begin{itemize}
		\item La prima pressione del pulsante A deve avviare il cronometro 
		\item La seconda pressione del pulsante A deve fermare il cronometro e mostrare il tempo trascorso sul display in secondi
		\item Il display deve mostrare sempre lo stato attuale del sistema (simbolo attesa, simbolo misurazione, tempo trascorso)
	\end{itemize}
\end{frame}

\begin{frame}
	\frametitle{Attività 4 - Reaction Game 1}
	Realizzare un programma che misuri il tempo di reazione di un giocatore.
	\vspace{2em}
	\begin{enumerate}
		\item Mostrare un simbolo di attesa
		\item Attende un tempo casuale tra 1 e 10 secondi
		\item Mostrare un simbolo di conferma sul display
		\item Misurare quanto tempo passa tra la comparsa del simbolo e la pressione del pulsante A da parte dell'utente
		\item Mostrare il tempo di reazione sul display in millisecondi
		\item (EXTRA) Impedire che il giocatore possa barare premendo il pulsante A prima che il simbolo di conferma sia apparso
	\end{enumerate}	
\end{frame}

\begin{frame}
	\frametitle{Attività 5 - Reaction Game 2}
	Estendere il programma della slide precedente introducendo un secondo giocatore.\\
	
	\vspace{0.5em}
	\begin{enumerate}
		\item Mostrare un simbolo di attesa
		\item Attende un tempo casuale tra 1 e 10 secondi
		\item Mostrare un simbolo sul display
		\item Attende che almeno uno dei due giocatori prema il proprio pulsante
		\item Misurare quanto tempo passa tra la comparsa del simbolo e la pressione di uno dei due pulsanti
		\item Mostrare il giocatore che ha vinto sul display
		\item (EXTRA) Impedire che i giocatori possa barare premendo il pulsante prima che il simbolo di conferma sia apparso
		\item (EXTRA) Mostrare anche lo scarto tra i tempi dei due giocatori (differenza tra i due).
	\end{enumerate}	
\end{frame}

\begin{frame}
	\frametitle{References}
	\begin{itemize}
		\item \href{https://microbit.org/get-started/user-guide/overview/}{Micro:Bit Scheme}
	\end{itemize}
\end{frame}
\end{document}