\documentclass{beamer}
\usepackage[utf8]{inputenc}
\usepackage[italian]{babel}
%Information to be included in the title page:
\title{Micro:Bit}

\begin{document}

\frame{\titlepage}
\begin{frame}
	\frametitle{Indice}
	\tableofcontents
\end{frame}
\begin{frame}
	\frametitle{Introduzione}
	This is some text in 
	the first frame. This is some text in the first frame. This is some text in the first frame.
\end{frame}

\begin{frame}
	\frametitle{Pulsanti}
	
\end{frame}

\begin{frame}
	\frametitle{Esercizio 1 - visualizzazione pressione pulsanti}
	Realizzare un programma che, ogni volta che si preme un pulsante, visualizzi sullo schermo il nome del pulsante premuto e il numero di volte che è stato premuto.\\
\end{frame}

\begin{frame}
	\frametitle{Misurazione del tempo}

\end{frame}

\begin{frame}
	\frametitle{Esercizio 2 - Conta secondi}
	Realizzare un programma che conta i secondi trascorsi da quando il micro:bit è stato acceso.\\
\end{frame}

\begin{frame}
	\frametitle{Esercizio 3 - Cronometro}
	Realizzare un cronometro che funzioni come segue:
	\begin{itemize}
		\item Premendo il pulsante A il cronometro parte o si ferma
		\item Premendo il pulsante B il cronometro mostra il tempo trascorso sul display
	\end{itemize}
\end{frame}

\begin{frame}
	\frametitle{Esercizio 4 - Tempi di reazione 1}
	Realizzare un programma che misuri il tempo di reazione di un giocatore.\\
	Il programma deve:
	\begin{enumerate}
		\item Attende un tempo casuale tra 1 e 10 secondi
		\item Mostrare un simbolo sul display
		\item Misurare quanto tempo passa tra la comparsa del simbolo e la pressione del pulsante A da parte dell'utente
		\item Mostrare il tempo di reazione sul display
	\end{enumerate}	
\end{frame}
\begin{frame}
	\frametitle{Esercizio 5 - Tempi di reazione 2}
	Estendere il programma della slide precedente introducendo un secondo giocatore.\\
	Il programma deve:
	\begin{enumerate}
		\item Attende un tempo casuale tra 1 e 10 secondi
		\item Mostrare un simbolo sul display
		\item Attende che almeno uno dei due giocatori prema il proprio pulsante
		\item Misurare quanto tempo passa tra la comparsa del simbolo e la pressione di uno dei due pulsanti
		\item Mostrare il giocatore che ha vinto e il tempo di reazione sul display
	\end{enumerate}	
\end{frame}
\end{document}