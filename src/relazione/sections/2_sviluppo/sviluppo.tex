\documentclass[../../relazione.tex]{subfiles}

\begin{document}
\section{Sviluppo dei contenuti}

\subsection{Materiale per studenti}
Per gli studenti è stata preparata una dispensa (\textit{studenti.pdf}) sotto forma di presentazione che accompagna le attività riassumendone ed elencandone le richieste e cartatteristiche.

Per un progetto è stato inoltre preparato un codice di partenza opzionale per semplificarne lo sviluppo.

\subsection{Guida per docenti}
Per i docenti è stata preparata una dispensa (\textit{docenti.pdf}) che spiega nel dettaglio le attività da svolgere, fornendo buona parte delle informazioni necessarie per presentare le attività senza bisogno di studiare materiale esterno.\\
Ogni attività è corredata dalla relativa soluzione (spesso anche più di una) e da una slide riassuntiva nella presentazione per gli studenti che può essere proiettata durante la spiegazione.

Le attività sono progettate per essere svolte in ordine, in modo da fornire in maniera incrementale le competenze necessarie per svolgere le attività successive. Alcune attività comprendono anche delle discussioni per stimolare gli studenti a ragionare sul codice e sulle soluzioni proposte.

Infine sono stati proposti tre progetti tra cui gli studenti possono scegliere e che devono essere implementati individualmente per casa. Per questi progetti è stata fornita una griglia di valutazione Rubric.
\end{document}