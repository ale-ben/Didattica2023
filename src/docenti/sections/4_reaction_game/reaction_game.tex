\documentclass[../../docenti.tex]{subfiles}

\begin{document}
\section{Reaction Game}
L'obiettivo è quello di realizzare un gioco in cui si valutano i riflessi dei giocatori.

Il gioco prevederà che due giocatori impugnino il MicroBit rispettivamente sul tasto A e B.
Il microcontrollore fornirà un sengale visivo e che indicherà ai giocatori di premere il tasto, il primo giocatore a premerlo vince.

Il gioco è semplice, ma per implementarlo è necessaria una buona comprensione di come funzionino variabili e cicli.

\customcallout{1em}{40em}{Per favorire la comprensione del progetto, si consiglia di sviluppare il lavoro in maniera progressiva, facendo realizzare prima una versione a giocatore singolo e poi una a due giocatori (come illustrato in \autoref{sec:reaction_game_1} e \autoref{sec:reaction_game_2}).}
\subsection{Prima implementazione}
\label{sec:reaction_game_1}
L'obiettivo è realizzare un misuratore di tempi di reazione che misuri il tempo che passa tra l'emissione di un segnale dal microcontrollore e la pressione del tasto A da parte del giocatore.

\subsubsection{[Attività] Identificazione delle 3 fasi principali del programma}
\customcallout{1em}{40em}{Si consiglia di impostare questa attività come discussione con tutta la classe, in modo da favorire la comprensione del progetto e la partecipazione di tutti gli studenti.}
Chiedere agli studenti di identificare le tre fasi principali del programma e descrivere cosa deve fare ognuna di queste.\\
Le tre fasi sono:
\begin{enumerate}
	\item \textbf{Attesa / Preparazione}
	\begin{itemize}
		\item Mostrare un segnale visivo di attesa (ad esempio una croce);
		\item Attendere un intervallo randomico di tempo;
	\end{itemize}
	\item \textbf{Misurazione}
	\begin{itemize}
		\item Mostrare un segnale visivo di conferma (ad esempio una spunta o un cerchio);
		\item Iniziare a misurare il tempo;
		\item Attendere la pressione del tasto A;
	\end{itemize}
	\item \textbf{Visualizzazione}
	\begin{itemize}
		\item Calcolare il tempo impiegato dal giocatore;
		\item Mostrarlo sul display;
	\end{itemize}
\end{enumerate}

\subsubsection{Attività 4 - Reaction Game 1}

La prima implementazione deve funzionare come segue:
\begin{enumerate}
	\item Il giocatore impugna il MicroBit sul tasto A;
	\item Il dislay del MicroBit mostra un segnale visivo di attesa (ad esempio una croce);
	\item Dopo un numero variabile di secondi (in un intervallo scelto dallo studente) il display mostra un segnale visivo di conferma (ad esempio una spunta o un cerchio) ed avvia un timer;
	\item Nel momento in cui il giocatore vede il segnale visivo di conferma, deve premere il tasto A; appena viene rilevata la pressione del tasto, il microcontrollore deve fermare il timer e mostrare il tempo impiegato dal giocatore in millisecondi.
	\item EXTRA: Impedire che il giocatore possa barare premendo il tasto prima che venga mostrato il segnale visivo di conferma.
\end{enumerate}

In questa prima implementazione, il giocatore gioca da solo e non è prevista una competizione, l'obiettivo è ottenere un tempo più basso possibile.

È stato inserito un obiettivo extra per permettere agli studenti più veloci di approfondire l'argomento e fornire più tempo agli studenti che possono averne bisogno.

\paragraph{Soluzione}
La soluzione può essere trovata nei file\\
 \textit{microbit-3-reaction\_game-1p.hex} e \textit{microbit-3-reaction\_game-1p-extra.hex}

\subsection{Seconda Implementazione}
\label{sec:reaction_game_2}

Questa implementazione sarà una modifica della prima, in cui verrà aggiunto un secondo giocatore.\\
L'obiettivo è realizzare un gioco in cui due giocatori si sfidano per vedere chi ha i riflessi migliori.

\subsubsection{[Attività] Identificazione delle variazioni rispetto alla prima implementazione}
Le differenze si trovano in fase di misurazione e visualizzazione.
\begin{enumerate}
	\item \textbf{Misurazione}
	\begin{itemize}
		\item Attendere la pressione del tasto A o del tasto B;
	\end{itemize}
	\item \textbf{Visualizzazione}
	\begin{itemize}
		\item Calcolare quale giocatore ha premuto il tasto per primo;
		\item Visualizzare A o B sul display per indicare il vincitore;
	\end{itemize}
\end{enumerate}

\subsubsection{Attività 5 - Reaction Game 2}
La seconda fase estende la prima, aggiungendo la possibilità di giocare in due utilizzando tasto A e tasto B.\\
La visuazzazione del tempo impiegato verrà sostituita con la visualizzazione del giocatore che ha premuto per primo il tasto.

L'implementazione deve funzionare come segue:
\begin{enumerate}
	\item I giocatori impugnano il MicroBit sul tasto A e B;
	\item Il dislay del MicroBit mostra un segnale visivo di attesa (ad esempio una croce);
	\item Dopo un numero variabile di secondi (in un intervallo scelto dallo studente) il display mostra un segnale visivo di conferma (ad esempio una spunta o un cerchio) ed avvia un timer;
	\item Nel momento in cui i giocatori vedono il segnale visivo di conferma, devono premere il tasto; appena viene rilevata la pressione del tasto, il microcontrollore salva il tempo di stop per quel tasto;
	\item Il microcontrollore confronta i due tempi e mostra sul display il giocatore che ha premuto per primo il tasto.
	\item EXTRA: Impedire che il giocatore possa barare premendo il tasto prima che venga mostrato il segnale visivo di conferma.
	\item EXTRA: Mostrare anche lo scarto tra i tempi dei due giocatori (differenza tra i due).
\end{enumerate}

\paragraph{[Attività] Discussione su possibili soluzioni alternative}
Il paragrafo \textit{Soluzione} presenterà due soluzioni, la soluzione A è quella attesa dato l'esercizio, la soluzione B è una soluzione alternativa che non usa il concetto di tempo.

\customcallout{1em}{40em}{Una volta ottenuta in aula la soluzione A, chiedere se gli studenti riescono a pensare ad una soluzione alternativa che non usi il concetto di tempo.}

Valutare vantaggi delle due soluzioni:
\begin{itemize}
	\item \textbf{A}
	\begin{itemize}
		\item Facilmente espandibile per mostrare anche il tempo impiegato (o lo scarto tra i due tempi);
	\end{itemize}
	\item \textbf{B}
	\begin{itemize}
		\item Minor numero di variabili (numero di variabili non dipendente dal numero di giocatori);
		\item Codice di più semplice comprensione;
	\end{itemize}
\end{itemize}

\paragraph{Soluzione}
Sono state identificate due soluzioni possibili, salvate nei file \\
\textit{microbit-4-reaction\_game-2p-a.hex} e \textit{microbit-4-reaction\_game-2p-b.hex}.

La prima soluzione è quella attesa, ovvero un evoluzione della prima implementazione in cui si misurano i tempi di reazione di entrambi i giocatori e si sceglie il giocatore con  il tempo minore.

La seconda soluzione, invece, rimuove completamente il concetto di tempo, tenendo traccia direttamente di quale pulsante è stato premuto per primo.\\
Per quanto non sia un evoluzione della prima implementazione, è comunque valida e può essere utilizzata per mostrare agli studenti come esistano più soluzioni per un problema.

\end{document}