\documentclass[../../docenti.tex]{subfiles}

\begin{document}
\section{Strumenti Utili}

\subsection{Risorse MicroBit Foundation}
MicroBit Foundation offre molte risorse utili per l'insegnamento della programmazione con MicroBit.\\
Tutte le risorse sono disponibili al seguente indirizzo: \url{https://microbit.org/teach}.

\subsection{MakeCode}
Come detto nelle sezioni precedenti, MakeCode è un editor online che permette di programmare MicroBit in modo semplice e intuitivo.

\subsubsection{Salvataggio e caricamento}
\label{sec:makecode_save_load}
Il codice creato con MakeCode può essere salvato in locale (tramite l'icona di salvataggio in basso o il pulsante \textit{Scarica}), questo genererà un file \textit{.hex}.\\
In qualunque momento si può trascinare un file \textit{.hex} nell'editor per caricare il codice salvato oppure si può usare il pulsante \textit{Importa} nella home di MakeCode.

\paragraph{Caricamento su MicroBit}
Per caricare il codice su MicroBit è necessario collegare la scheda al computer tramite cavo USB e trascinare il file \textit{.hex} (vedere paragrafo sopra) nel dispositivo \textit{MICROBIT} che apparirà sul computer.

\subsubsection{MicroBit Classroom}
MicroBit Classroom (\url{https://classroom.microbit.org}) è un servizio online che permette\\ di creare classi virtuali (in ambiente Make:Code) con gli studenti e monitorare in tempo reale i loro progressi.

Questo strumento è molto utile poichè consente di condividere il proprio editor con gli studenti mentre si spiega e vedere tutti i loro editor mentre svolgono esercizi (è inoltre possibile scaricare il codice degli studenti per una successiva revisione).

La piattaforma non richiede alcuna registrazione e può essere utilizzata da chiunque.

\end{document}