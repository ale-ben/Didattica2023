\documentclass[../../docenti.tex]{subfiles}

\begin{document}
\section{Livella}
La livella (comunemente nota come \textit{bolla}) è uno strumento utilizzato per verificare la planarità di una superficie.\\
Micro:Bit è dotato di giroscopio, sensore che consente di rilevare l'orientamento del dispositivo nello spazio.

L'obiettivo di questa sezione sarà quello di sviluppare una livella digitale utilizzando il giroscopio di Micro:Bit e il display.

\subsection{Rollio e Beccheggio}
L'inclinazione lungo i due assi del giroscopio è chiamata \textit{rollio} (\textit{roll}) e \textit{beccheggio} (\textit{pitch}).

Si parla di rollio quando il dispositivo ruota lungo l'asse verticale del dispositivo (inclinato verso destra o sinistra).\\
Si parla di beccheggio quando il dispositivo ruota lungo l'asse orizzontale del dispositivo (inclinato in avanti o indietro).

Micro:Bit rappresenta rollio e beccheggio come valori numerici compresi tra -180 e 180 gradi, dove il valore 0 rappresenta lo stato "piano" rispetto all'asse.

Il comando per accedere ai dati del giroscopio si trova in \textit{Ingressi -- altro} e si chiama \textit{rotazione}.

\subsection{Griglia LED}
Fino ad ora la griglia LED è sempre stata utilizzata per disegnare immagini o scritte.\\
La successiva attività richiederà il controllo dei singoli LED per rappresentare l'inclinazione del dispositivo.

Si può accendere un singolo LED della griglia utilizzando il comando \textit{disegna} presente nel pannello \textit{LED} e specificando le coordinate del LED da accendere.\\
Notare che la griglia è composta da 5x5 LED e che le coordinate vanno da 0 a 4, dove (0, 0) si trova in alto a sinistra.
\newpage
\subsection{Attività 6 - Livella}
L'obiettivo di questa attività è quello di sviluppare una livella digitale che mostri all'utente l'inclinazione del dispositivo lungo i due assi sfruttando la griglia di LED.

L'implementazione deve funzionare come segue:
\begin{itemize}
	\item Il dispositivo deve misurare l'inclinazione lungo i due assi e salvarla in apposite variabili;
	\item Tramite la formula scritta di seguito il programma deve determinare le coordinate del LED da accendere per rappresentare l'inclinazione lungo i due assi;
	\item Il programma deve pulire lo schermo e accedere il LED calcolato al punto precedente;
	\item (EXTRA) Tramite la pressione dei pulsanti A o B l'utente deve poter scegliere di focalizzarsi solo su uno dei due assi.
	\begin{itemize}
		\item Il pulsante A deve visualizzare solo le variazioni in beccheggio;
		\item Il pulsante B deve visualizzare solo le variazioni in rollio;
	\end{itemize}
\end{itemize}

\paragraph{Calcolo delle coordinate}
La coordinata del LED da accendere può essere calcolata partendo dall'angolo \(\alpha\) per casi come segue:
\begin{center}
	\begin{math}
		\begin{cases}
			0\quad\text{se } \alpha > 20\\
			1\quad\text{se } \alpha > 10\\
			2\quad\text{se } -10 \leq \alpha \geq 10\\
			3\quad\text{se } \alpha < -10\\
			4\quad\text{se } \alpha < -20
		\end{cases}
	\end{math}
\end{center}

\subparagraph{Suggerimenti}
É consigliato implementare la conversione da angolo a coordinata in una funzione a parte per rendere il programma più leggibile ed evitare ripetizioni di codice.

\paragraph{Soluzione}
La soluzione può essere trovata nei file\\
\textit{microbit-5.livella.hex} e \textit{microbit-5.livella-extra.hex}.

\end{document}